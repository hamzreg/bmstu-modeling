\chapter{Задание}

Реализовать программу с графическим интерфейсом для моделирования процесса обработки 400 заявок абитуриентов приемной комиссией факультета ИУ МГТУ им. Н.~Э. Баумана и определения числа абитуриентов, подавших документы на кафедры ИУ1-ИУ4, ИУ5-ИУ8 и ИУ9-ИУ12. Приемная комиссия работает следующим образом:

\begin{enumerate}
	\item Абитуриенты приходят через интервал времени, равный 10 $\pm$ 5 мин.
	\item Абитуриент заполняет анкету у одного из двух сотрудников. Сотрудники имеют разную производительность и могут обеспечивать обработку анкеты абитуриента за 25 $\pm$ 10 мин. и 15 $\pm$ 15 мин. соответственно. Абитуриенты стремятся занять свободного сотрудника с максимальной производительностью. Если оба сотрудника заняты, абитуриент подает документы в другой день.
	\item Абитуриент попадает в одну из трех очередей в зависимости от кафедры, студентом которой хочет стать абитуриент. С вероятностью 0.2 абитуриент выбирает одну из кафедр ИУ1-ИУ4, с вероятностью 0.5 --- одну из кафедр ИУ5-ИУ8, с вероятностью 0.3 --- одну из кафедр ИУ9-ИУ12.
	\item Три сотрудника принимают документы у абитуриентов из очередей следующим образом: первый сотрудник обслуживает абитуриентов из первой очереди за 15 $\pm$ 2 мин., второй сотрудник --- из второй очереди за 25 $\pm$ 10 мин., третий сотрудник --- из третьей очереди за 20 $\pm$ 5 мин.
\end{enumerate}

\section{Схемы модели}

На рисунке \ref{img:structural_scheme} показана структурная схема модели, а на рисунке \ref{img:qsystem_scheme} представлена схема модели в терминах СМО.

\includeimage
    {structural_scheme}
    {f}
    {h}
    {1.0\textwidth}
    {Структурная схема модели}

\includeimage
    {qsystem_scheme}
    {f}
    {h}
    {1.0\textwidth}
    {Схема модели в терминах СМО}

\chapter{Реализация}

\section{Детали реализации}

На листинге \ref{lst:enrollee_generator.py} показана реализация класса генератора абитуриентов.

\includelistingpretty
    {enrollee_generator.py}
    {Python}
    {Моделирование работы генератора абитуриентов}
\newpage

На листинге \ref{lst:handler.py} представлена реализация класса сотрудника, обрабатывающего анкеты.

\includelistingpretty
    {handler.py}
    {Python}
    {Моделирование работы сотрудника, обрабатывающего анкеты}
    
На листинге \ref{lst:receiver.py} показана реализация класса сотрудника, принимающего документы.

\includelistingpretty
    {receiver.py}
    {Python}
    {Моделирование работы сотрудника, принимающего документы}

На листинге \ref{lst:commission.py} представлена реализации класса приемной комиссии и функции выбора кафедры абитуриентом.

\includelistingpretty
    {commission.py}
    {Python}
    {Моделирование работы приемной комиссии и выбора кафедры абитуриентом}

\section{Полученный результат}

На рисунке \ref{img:commission} показана страница программы для определения числа абитуриентов, подавших документы на кафедры ИУ1-ИУ4, ИУ5-ИУ8 и ИУ9-ИУ12, при моделировании работы приемной комиссии с заданными в условии параметрами.

\includeimage
    {commission}
    {f}
    {h}
    {0.6\textwidth}
    {Моделирование работы приемной комиссии}


