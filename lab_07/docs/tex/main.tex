\chapter{Задание}

Реализовать программу для моделирования работы системы массового обслуживания и определения максимальной длины очереди, при которой не будет потери сообщений, на языке имитационного моделирования GPSS. Моделируемая система состоит из генератора сообщений, очереди сообщений и обслуживающего аппарата. Для моделирования работы генератора сообщений использовать равномерный закон распределения, для моделирования работы обслуживающего аппарата --- нормальный закон распределения. Предусмотреть возможность возврата в очередь части обработанных сообщений с заданной вероятностью.

\section{Закон появления сообщений}

Для моделирования работы генератора сообщений в лабораторной работе используется равномерный закон распределения. Случайная величина имеет равномерное распределение на отрезке $[a, b]$, если её функция плотности $p(x)$ имеет вид:

\begin{equation}
    p(x) = 
    \begin{cases}
        \frac{1}{b - a}, \text{если } x \in [a, b],\\
        0, \text{иначе.} \\
    \end{cases}
\end{equation}

Функция распределения $F(x)$ равномерной случайной величины имеет вид:

\begin{equation}
    F(x) = 
    \begin{cases}
    		0, \text{если } x \leqslant a, \\
        \frac{x - a}{b - a}, \text{если } a < x \leqslant b,\\
        1, \text{если } x > b. \\
    \end{cases}
\end{equation}

Интервал времени между появлением $i$-ого и $(i - 1)$-ого сообщения по равномерному закону распределения вычисляется следующим образом:

\begin{equation}
	T_{i} = a + (b - a) \cdot R,
\end{equation}

\noindentгде $R$ --- псевдослучайное число от 0 до 1.

\section{Закон обработки сообщений}

Для моделирования работы генератора сообщений в лабораторной работе используется нормальный закон распределения. Случайная величина имеет нормальное распределение, если её функция плотности $p(x)$ имеет вид:

\begin{equation}
    p(x) = \frac{1}{\sigma \cdot \sqrt{2 \cdot \pi}} \cdot e^{-\frac{(x - \mu)^2}{2 \cdot \sigma^2}}, (-\infty < \mu < +\infty, \sigma > 0).
\end{equation}

Функция распределения $F(x)$ нормальной случайной величины имеет вид:

\begin{equation}
    F(x) = \frac{1}{2} \cdot (1 + erf(\frac{x - \mu}{\sqrt{2 \cdot \sigma^2}})).
\end{equation}

Интервал времени между появлением $i$-ого и $(i - 1)$-ого сообщения по нормальному закону распределения вычисляется следующим образом:

\begin{equation}
	T_{i} = \sigma_{X} \cdot \sqrt{\frac{12}{n}} \cdot (\sum_{i = 1}^n R_{i} - \frac{n}{2}) + M_{X}, 
\end{equation}

\noindentгде $n = 12$, $R_{i}$ --- псевдослучайное число от 0 до 1.

\chapter{Реализация}

\section{Детали реализации}

На листинге \ref{lst:qsystem.txt} представлена реализация работы системы массового обслуживания.

\includelisting
    {qsystem.txt}
    {Реализация работы системы массового обслуживания}
\newpage

\section{Полученный результат}

На листинге \ref{lst:report.txt} показан результат моделирования работы системы массового обслуживания. Максимальная длина очереди для вероятности возврата сообщения 0.3 равна 150 сообщениям.

\includelisting
    {report.txt}
    {Реализация работы системы массового обслуживания}
\newpage

