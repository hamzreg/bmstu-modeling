\chapter{Задание}

Реализовать программу с графическим интерфейсом для моделирования процесса обработки 300 запросов клиентов информационным центром и определения вероятности отказа клиенту в обслуживании. Информационный центр работает следующим образом:

\begin{enumerate}
	\item Клиенты приходят через интервал времени, равный 10 $\pm$ 2 мин.
	\item Если все три имеющихся оператора заняты, клиенту отказывают в обслуживании. Операторы имеют разную производительность и могут обеспечивать обслуживание среднего запроса пользователя за 20 $\pm$ 5 мин., 40 $\pm$ 10 мин. и 40 $\pm$ 20 мин. соответственно. Клиенты стремятся занять свободного оператора с максимальной производительностью.
	\item Полученные запросы сдаются в приемный накопитель, из которого они выбираются для обработки. На первый компьютер выбираются запросы от первого и второго операторов, на второй компьютер --- от третьего оператора. Время обработки на первом и втором компьютерах равны соответственно 15 мин. и 30 мин.
\end{enumerate}

В процессе взаимодействия клиентов и информационного центра возможны: режим нормального обслуживания, когда клиент выбирает одного из свободных операторов c максимальной производительностью, и режим отказа.

\section{Моделирование функционирования системы}

При моделировании функционирования системы эндогенными переменными являются:

\begin{itemize}
	\item время обслуживания клиента $i$-ым оператором, где $i = \overline{1, 3}$;
	\item время обработки запроса на $j$-ом компьютере, где $j = \overline{1, 2}$.
\end{itemize}

Экзогенными переменными являются:

\begin{itemize}
	\item число обслуженных клиентов $n_{0}$;
	\item число клиентов, получивших отказ, $n_{1}$.
\end{itemize}

Уравнение модели имеет следующий вид:

\begin{equation}
    P_{\text{отказа}} = \frac{n_{1}}{n_{0} + n_{1}}
\end{equation}

\section{Структурная схема модели}

На рисунке \ref{img:structural_scheme} показана структурная схема модели.

\includeimage
    {structural_scheme}
    {f}
    {h}
    {1.0\textwidth}
    {Структурная схема модели}

\section{Схема модели в терминах СМО}

На рисунке \ref{img:qsystem_scheme} представлена схема модели в терминах СМО.

\includeimage
    {qsystem_scheme}
    {f}
    {h}
    {1.0\textwidth}
    {Схема модели в терминах СМО}

\chapter{Реализация}

\section{Детали реализации}

На листинге \ref{lst:client_generator.py} показана реализация класса генератора клиентов.

\includelistingpretty
    {client_generator.py}
    {Python}
    {Моделирование работы генератора клиентов}
\newpage

На листинге \ref{lst:operator.py} представлена реализация класса оператора.

\includelistingpretty
    {operator.py}
    {Python}
    {Моделирование работы оператора}
    
На листинге \ref{lst:computer.py} показана реализация класса компьютера.

\includelistingpretty
    {computer.py}
    {Python}
    {Моделирование работы компьютера}

На листинге \ref{lst:center.py} представлена реализация класса информационного центра.

\includelistingpretty
    {center.py}
    {Python}
    {Моделирование работы информационного центра}

\section{Полученный результат}

На рисунке \ref{img:center} показана страница программы для определения вероятности отказа клиенту при моделировании процесса обработки запросов клиентов информационным центром с заданными в условии параметрами.

\includeimage
    {center}
    {f}
    {h}
    {0.6\textwidth}
    {Моделирование работы информационного центра}


