\chapter{Задание}

Реализовать программу с графическим интерфейсом для моделирования работы системы массового обслуживания принципом $\Delta t$ и событийным принципом и определения максимальной длины очереди, при которой не будет потери сообщений. Моделируемая система состоит из генератора сообщений, очереди сообщений и обслуживающего аппарата. Для моделирования работы генератора сообщений использовать равномерный закон распределения, для моделирования работы обслуживающего аппарата --- нормальный закон распределения. Предусмотреть возможность возврата в очередь части обработанных сообщений с заданной вероятностью.

\section{Принцип $\Delta t$}

Принцип $\Delta t$ заключается в последовательном анализе состояний всех блоков в момент $t + \Delta t$ по заданному состоянию блоков в момент времени $t$. При этом новое состояние блоков объявляется в соответствии с их алгоритмическим описанием с учетом действующих случайных факторов, которые задаются распределениями вероятностей. В результате анализа принимается решение о том, какие общесистемные события должны имитироваться программой на данный момент времени.

\section{Событийный принцип}

При использовании событийного принципа состояния всех блоков имитационной модели анализируется лишь в момент появления какого-либо события. Момент поступления следующего события определяется минимальным значением из списка будущих событий, представляющего собой совокупность моментов ближайшего изменения состояний каждого из блоков системы.

\section{Закон появления сообщений}

Для моделирования работы генератора сообщений в лабораторной работе используется равномерный закон распределения. Случайная величина имеет равномерное распределение на отрезке $[a, b]$, если её функция плотности $p(x)$ имеет вид:

\begin{equation}
    p(x) = 
    \begin{cases}
        \frac{1}{b - a}, \text{если } x \in [a, b],\\
        0, \text{иначе.} \\
    \end{cases}
\end{equation}

Функция распределения $F(x)$ равномерной случайной величины имеет вид:

\begin{equation}
    F(x) = 
    \begin{cases}
    		0, \text{если } x \leqslant a, \\
        \frac{x - a}{b - a}, \text{если } a < x \leqslant b,\\
        1, \text{если } x > b. \\
    \end{cases}
\end{equation}

Интервал времени между появлением $i$-ого и $(i - 1)$-ого сообщения по равномерному закону распределения вычисляется следующим образом:

\begin{equation}
	T_{i} = a + (b - a) \cdot R,
\end{equation}

\noindentгде $R$ --- псевдослучайное число от 0 до 1.

\section{Закон обработки сообщений}

Для моделирования работы генератора сообщений в лабораторной работе используется нормальный закон распределения. Случайная величина имеет нормальное распределение, если её функция плотности $p(x)$ имеет вид:

\begin{equation}
    p(x) = \frac{1}{\sigma \cdot \sqrt{2 \cdot \pi}} \cdot e^{-\frac{(x - \mu)^2}{2 \cdot \sigma^2}}, (-\infty < \mu < +\infty, \sigma > 0).
\end{equation}

Функция распределения $F(x)$ нормальной случайной величины имеет вид:

\begin{equation}
    F(x) = \frac{1}{2} \cdot (1 + erf(\frac{x - \mu}{\sqrt{2 \cdot \sigma^2}})).
\end{equation}

Интервал времени между появлением $i$-ого и $(i - 1)$-ого сообщения по нормальному закону распределения вычисляется следующим образом:

\begin{equation}
	T_{i} = \sigma_{X} \cdot \sqrt{\frac{12}{n}} \cdot (\sum_{i = 1}^n R_{i} - \frac{n}{2}) + M_{X}, 
\end{equation}

\noindentгде $n = 12$, $R_{i}$ --- псевдослучайное число от 0 до 1.

\chapter{Реализация}

\section{Детали реализации}

На листинге \ref{lst:step.py} показана реализация функции управляющей программы принципом $\Delta t$.

\includelistingpretty
    {step.py}
    {Python}
    {Реализация управляющей программы принципом $\Delta t$}
\newpage

На листинге \ref{lst:eventful.py} представлена реализация функции управляющей программы событийным принципом.

\includelistingpretty
    {eventful.py}
    {Python}
    {Реализация управляющей программы событийным принципом}
    
На листинге \ref{lst:uniform.py} показана реализация функции вычисления интервала времени между появлением $i$-ого и $(i - 1)$-ого сообщения по равномерному закону распределения.

\includelistingpretty
    {uniform.py}
    {Python}
    {Вычисление интервала времени между появлениями сообщений по равномерному закону распределения}

На листинге \ref{lst:normal.py} представлена реализация функции вычисления интервала времени между появлением $i$-ого и $(i - 1)$-ого сообщения по нормальному закону распределения.

\includelistingpretty
    {normal.py}
    {Python}
    {Вычисление интервала времени между появлениями сообщений по нормальному закону распределения}

\section{Полученный результат}

На рисунках \ref{img:zero}-\ref{img:one} показаны страницы программы для определения максимальной длины очереди при моделировании системы массового обслуживания принципом $\Delta t$ и событийным принципом с вероятностями возврата сообщения 0.0, 0.3, 0.7, 1.0 соответственно.

\includeimage
    {zero}
    {f}
    {h}
    {0.6\textwidth}
    {Моделирование системы массового обслуживания с вероятностью возврата сообщения 0.0}
    
\includeimage
    {three}
    {f}
    {h}
    {0.6\textwidth}
    {Моделирование системы массового обслуживания с вероятностью возврата сообщения 0.3}
    
\includeimage
    {seven}
    {f}
    {h}
    {0.6\textwidth}
    {Моделирование системы массового обслуживания с вероятностью возврата сообщения 0.7}
    
\includeimage
    {one}
    {f}
    {h}
    {0.6\textwidth}
    {Моделирование системы массового обслуживания с вероятностью возврата сообщения 1.0}

