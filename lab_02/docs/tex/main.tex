\chapter{Задание}

Реализовать программу с графическим интерфейсом для определения времени пребывания системы в каждом состоянии и предельных вероятностей в установившемся режиме работы системы массового обслуживания. Максимальное количество состояний системы равно десяти.

\section{Марковский процесс}

Случайный процесс, протекающий в некоторой системе $S$ называется марковским, если он обладает следующим свойством: для каждого момента времени вероятность любого состояния системы в будущем зависит только от ее состояния в настоящем и не зависит от того, когда и каким образом система пришла в это состояние.

Для марковского процесса составляют систему уравнений Колмогорова по следующему правилу: в левой части каждого уравнения стоит производная вероятности $i$-го состояния, в правой части --- сумма произведений вероятностей всех состояний, приводящих систему в данное состояние, на интенсивности соответствующих переходов, минус суммарная интенсивность всех переходов, выводящих систему из данного состояния, умноженная на вероятность данного состояния.

\section{Определение предельных вероятностей}

Для определения предельных вероятностей в составленной системе уравнений Колмогорова производные вероятностей состояний заменяются нулевыми значениями, и одно из уравнений заменяется уравнением нормировки: $\sum_{i = 1}^{n}p_{i}(t) = 1$, где $n$ --- количество состояний системы.

\section{Определение точек стабилизации}

Для определения точек стабилизации системы определяются вероятности состояний в моменты времени $t$ с малым шагом $\Delta t$. Для точки стабилизации должно выполняться условие:

\begin{equation}
	|\lim_{t \to \infty}p_{i}(t) - p_{i}(t)| < \varepsilon, \text{где } \varepsilon \text{ - заданная точность.}
\end{equation} 

\chapter{Реализация}

\section{Детали реализации}

На листинге \ref{lst:probabilities.py} показаны реализации функции определения коэффициентов уравнений Колмогорова и функции определения предельных вероятностей.

\includelistingpretty
    {probabilities.py}
    {Python}
    {Определение коэффициентов уравнений Колмогорова и предельных вероятностей}
    
На листинге \ref{lst:time.py} представлены реализации функции определения производных и функции определения точек стабилизации.

\includelistingpretty
    {time.py}
    {Python}
    {Определение производных и точек стабилизации}

\section{Полученный результат}

На рисунках \ref{img:qsystem_four}-\ref{img:qsystem_four_char} и \ref{img:qsystem_five}-\ref{img:qsystem_five_char} представлены страницы программы для определения времени пребывания системы в каждом состоянии и предельных вероятностей в установившемся режиме работы систем массового обслуживания, состоящих из четырех и пяти состояний соответственно.

\includeimage
    {qsystem_four}
    {f}
    {h}
    {1.0\textwidth}
    {Система массового обслуживания, состоящая из четырех состояний}
    
\includeimage
    {qsystem_four_char}
    {f}
    {h}
    {0.8\textwidth}
    {График зависимости вероятности от времени для системы, состоящей из четырех состояний}
    
\includeimage
    {qsystem_five}
    {f}
    {h}
    {1.0\textwidth}
    {Система массового обслуживания, состоящая из пяти состояний}
    
\includeimage
    {qsystem_five_char}
    {f}
    {h}
    {0.8\textwidth}
    {График зависимости вероятности от времени для системы, состоящей из пяти состояний}
